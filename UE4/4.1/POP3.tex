%%
%% Author: thompson
%% 09.11.17
%%

% Preamble
\documentclass[11pt]{article}

% Packages
\usepackage{a4wide}

% Document
\begin{document}
    \section{Post Office Protocol 3}
    .. anhand RFC 1939.
    \subsection{Wie ist das POP3-Protokoll grundsätzlich aufgebaut?}
    The Post Office Protocol - Version 3 (POP3) is intended to
    permit a workstation to dynamically access a maildrop on a server
    host in a useful fashion.  Usually, this means that the POP3 protocol
    is used to allow a workstation to retrieve mail that the server is
    holding for it. POP3 is not intended to provide extensive manipulation operations of
    mail on the server - normally, mail is downloaded and then deleted.

    Initially, the server host starts the POP3 service by listening on
    TCP port 110. When a client host wishes to make use of the service,
    it establishes a TCP connection with the server host.  When the
    connection is established, the POP3 server sends a greeting.  The
    client and POP3 server then exchange commands and responses
    (respectively) until the connection is closed or aborted.

    Commands in the POP3 consist of a case-insensitive keyword, possibly
    followed by one or more arguments.  All commands are terminated by a
    CRLF pair.  Keywords and arguments consist of printable ASCII
    characters.  Keywords and arguments are each separated by a single
    SPACE character.  Keywords are three or four characters long. Each
    argument may be up to 40 characters long.

    Responses in the POP3 consist of a status indicator and a keyword
    possibly followed by additional information.  All responses are
    terminated by a CRLF pair.  Responses may be up to 512 characters
    long, including the terminating CRLF.  There are currently two status
    indicators: positive ("+OK") and negative ("-ERR").  Servers MUST
    send the "+OK" and "-ERR" in upper case. Responses to certain commands are multi-line.

    Sessions progresses through a number of states during its
    lifetime.  Once the TCP connection has been opened and the POP3
    server has sent the greeting, the session enters the AUTHORIZATION
    state.  In this state, the client must identify itself to the POP3
    server.  Once the client has successfully done this, the server
    acquires resources associated with the client's maildrop, and the
    session enters the TRANSACTION state.  In this state, the client
    requests actions on the part of the POP3 server.  When the client has
    issued the QUIT command, the session enters the UPDATE state.  In
    this state, the POP3 server releases any resources acquired during
    the TRANSACTION state and says goodbye.  The TCP connection is then
    closed.


    \subsection{Ist POP3 ein sicheres Protokoll? Warum, Warum nicht?}

    \subsection{Was ist der Unterschied zwischen single-line und multi-line response?}

    \subsection{Vergleiche POP3 mit IMAP und beschreibe wesentliche Unterschiede.}


\end{document}