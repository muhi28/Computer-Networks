%%
%% Author: thompson
%% 26.10.17
%%

% Preamble
\documentclass[11pt]{article}

% Packages
\usepackage{a4wide}
\usepackage[utf8]{inputenc}     % Enables ö ä ü
\usepackage[ngerman]{babel}     % Deutsche Rechtschreibung @ Long words

% Document
\begin{document}
    \section{Internetarchitekturen}
    Die allgemeine Architektur des Internets ist gegeben durch die Vernetzung der
    Systeme auf unterschiedlichen Ebenen.  Nutzer können über Internet Exchange Points
    ("Internetknoten", IX or IXP, auch Network Access Point) verschiedenste Daten
    austauschen und generell werden diese von Providern (ISPs) zur Nutzung dargeboten.

    Die meisten ISPs nutzen IX als Schnittstellen zwischen Rechnernetzwerken, wobei der
    gesamte Verbund aller autonomen Systeme das Internet bilden. Weltweit existieren ca.
    340 IXPs, wobei kleinere Knotenpunkte als Uplink zu den 'Carriern', den ISPs, dienen.
    Die Vorteile mehrerer IX sind primär Effizienz und Ausfallsicherheit bei Datentransfer, wobei
    die Kosten für den Betrieb eines IX von den dazugehörigen ISPs geteilt werden. Die Gebühren
    berechnen sich pro genutzten Port per eigenen IXP. Die Kosten für den jeweiligen Port sind
    abhängig von dessen Transferrate - derzeit zwischen 10Mbps und 100 Gbps.\\

    \noindent Bei den einzelnen ISPs unterscheidet man zwischen mehreren Kategorien (Tiers):
    \begin{itemize}
    \item Tier 1.: National \& oftmals International, die größten Betreiber. \\
    z.B.: Deutsche Telekom, KPN, AT\&T, Verizon, NTT, Telecom Italia,..
    \item Tier 2.: Transit Provider. Nehmen Downstream von Tier 1 in Anspruch und bieten Upstream für Tier 3.\\
    z.B.: Vodafone, Tele2, Comcast
    \item Tier 3.: Lokale Provider. Sie verkaufen Transitmöglichkeiten an Nutzer.
    \end{itemize}
    \noindent Es bleibt jedoch zu beachten dass die Kategorisierung regelrecht schwammig geführt wird.\\

    \noindent Weiters ist das Internet noch durch Protokolle unterstützt, um fehlerfreien Austausch
    zu garantieren, oft beschrieben mithilfe des ISO/OSI-Referenzmodells (siehe 2.3).

\end{document}