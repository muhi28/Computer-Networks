%%
%% Author: thompson
%% 03.11.17
%%

% Preamble
\documentclass[11pt]{article}

% Packages
\usepackage{a4wide}
\usepackage[ngerman]{babel}
\usepackage[utf8]{inputenc}

\usepackage{scrextend}
\usepackage{enumerate}
\usepackage{graphicx}
\usepackage{verbatim}

% Document
\begin{document}
    \section{Analyse mit Wireshark}
    Nun werden wir heutigen Datenverkehr aufnehmen und analysieren. Via Youtube catchen wir
    einige Datenpackete welche wir anschließend begutachten, gemäß folgender Fragen:

    \begin{enumerate}
        \item Versuchen Sie die Verbindung über unterschiedliche Zugangsnetzwerke (z.B. LAN, WLAN, 3/4G sofern möglich) herzustellen und dokumentieren Sie allfällige Unterschiede.\\
        Auffallend ist die unterschiedliche Anzahl an versendeten Packeten. Dies ist schätzungsweise rückführbar auf veränderte MMS.\\

        \item Welche Objekte werden vom Client via HTTP angefordert? Hinweis: nur jene beim Videostreaming, andere Objekte (z.B. HTML, Text, Bilder) können vernachlässigt werden.
            Wir setzen Wireshark auf unser Netzwerk an und starten  den Stream.
            Wir beenden die Aufnahme alsbald und analysieren unsere .pcapng's mithilfe von Filter, Packetuntersuchungen und
            eingesetztem PSK Key (welcher nicht so ganz funktionierte wie er soll).
        \begin{verbatim}
            https://www.wireshark.org/tools/wpa-psk.html
            Passphrase: PW
            SSID: Network
                -> Generate PSK
            Initiate WEP-Decryption
            > Edit > Preferences > Protocols ... IEEE 802.11: Enable Decryption
             ... Decryption Keys Edit > Add: WEP > Add PSK in empty ""
        \end{verbatim}
        \begin{enumerate}[\thesection .1]
            \item Stream eines Youtube-Videos via Ethernetverbindung\\
            Wenn wir Info betrachten, so können wir direkt den TCP-Handshake des Klienten und des Servers verifizieren, ebenso dass
            der Server uns nicht via RST abgewiesen hat.
            Ebenso können wir mittels Filter die genutzten Protokolle identifizieren: ARP,  SNMP, TCP und TSLv1.2, ICMP, DNS
            \begin{addmargin}[1em]{1em}
                TSLv1.2 beschreibt eine SSL-Enkryption vom TCP-Austausch über Secure Sockets (Spezielle Ports)
                SNMP als das Zuweisungsprotokoll innerhalb unseres Heimnetzwerkes
            \end{addmargin}
            Weiters können wir feststellen, dass es beim Verbindungsaufbau zur URL bis hin zur Beendigung des Videos insgesamt
            \begin{addmargin}[1em]{1em}
                - 6 unterschiedliche IPs miteinander kommunizierten und im Rahmen dieser
                \begin{addmargin}[1em]{1em}
                    - 19 unterschiedliche IPv4 Konversationen erfolgten\\
                    - 16 Konversationen über TCP verliefen\\
                    - 4 UDP Transfers stattfanden\\
                \end{addmargin}
            \end{addmargin}
            Ein Objektzugriff über HTTP erfolgte hier erstaunlicherweise nicht, was jedoch mit der Implementierung seitens Youtube zusammenhängt.\\

        Wir starten einen neuen Sniff-Versuch, nun jedoch über das WLan unseres Heimnetzwerkes.
        \item Stream eines Youtube-Videos via WLan-Verbindung\\
        Der Stream über WLan erfolgte über weniger Packeten als wie beim Stream durch direkte Ethernetverbindung.
        Die Protokolltypen blieben die gleichen, während es einen Umschwung bei den Konversationen gab.
        Diesmal kommunizierten lediglich
        \begin{addmargin}[1em]{1em}
            - 2 unterschiedliche IPs miteinander, samt
            \begin{addmargin}[1em]{1em}
                - 12 unterschiedlichen IPv4 Konversationen\\
                - 26 TCP Konversationen\\
                - und 1 UDP Transfer\\
            \end{addmargin}
        \end{addmargin}

            Bei beiden Versuchen handelte es sich um denselben Anfangszustand (Videoauswahl Youtube),
            dasselbe Video und dieselbe Videoqualität.\\

            Wir versuchen uns diesmal an einem anderen Anbieter:
            \item Stream von anilinkz.io\footnote[1]{http://anilinkz.to/no-game-no-life-episode-1?src=8} über Ethernet
            Wir umgehen hierbei triviale Daten und starten direkt mit dem Stream selbst, welcher jedoch wiederum Werbung hervorruft
            (dementsprechend die abgebrochenen TCP-Packete am Start).

            Über die Zeitdauer von 5 Minuten ergaben sich ein gutes Dutzend an Konversationen.
            Verteilt über
            \begin{addmargin}[1em]{1em}
                - 12 unterschiedliche IPs ergaben sich
                \begin{addmargin}[1em]{1em}
                    - 138 IPv4 und 5 IPv6 Konversationen\\
                    - 187 TCP Datentransfers und \\
                    - 9 UDP Übertragungen\\
                \end{addmargin}
            \end{addmargin}
            Der Hauptverbindungsträger ist www11.mp4upload.com, eine klassifizierte Subdomain eines Hosts, ein Hinweis auf implementiertes Server Load Balancing.

            Es erfolgten hierbei Konversationen über Protokolle wie TLS1.2, TCP, UDP, DNS, MDNS, ARP, IGMPv2 ...
            \begin{addmargin}[1em]{1em}
                - MDNS oder Multicast DNS dient der IP-Adressierungen in kleineren Netzwerken.
            \end{addmargin}
        \end{enumerate}
    \end{enumerate}
\end{document}