%%
%% Author: muhamed
%% 29.10.17
%%

% Preamble
\documentclass[11pt]{article}

% Packages
\usepackage{a4wide}
\usepackage[utf8]{inputenc}
\usepackage[ngerman]{babel}
\usepackage{scrextend}      % Intending
\usepackage{graphicx}

% Document
\begin{document}

    \section{Streaming}

    Grob gesagt meint Streaming das Abspielen von Inhalten über das Internet bzw. über ein Netzwerk.
    Hierbei werden während dem streamen fortlaufend Datenpakete übertragen und direkt verarbeitet.
    Würde man zum Beispiel ein Video auf Youtube ansehen, wird die Videodatei in kleinen Teilen auf das
    Gerät heruntergeladen und direkt abgespielt. Im Gegensatz zu einem herkömmlichen Download werden die
    Daten aber nicht dauerhaft gespeichert, sonden direkt wieder verworfen.
    \\\\Das Streaming wird jedoch dann Problematisch, wenn die Internetverbindung zu langsam ist, da in diesem
    Fall das Video nähmlich stark stocken würde. Der Grund dafür ist der, dass die nächsten Daten noch nicht
    heruntergeladen sind.
    Ein Stream kann natürlich auch von einem GErät auf ein anderes stattfinden. Beispielsweise können Medien von einem
    IPhone auf ein Apple TV gestreamt werden.\\\\

    \emph{Beispiele:}\\

    \begin{addmargin}[1em]{1em}

        \begin{enumerate}

            \item \emph{Netflix:}\\\\ Die Inhalte, also Serien und Filme, liegen auf den Netflix-Servern. Wird nun ein Video
            gestartet, so wird die Datei über ihre Internetverbindung auf ihre nPC gestreamt. Um das Video ohne Probleme
            anzusehen, ist eine Internetverbindung mit einer Geschwindigkeit von mindestens 6000 Mbits pro Sekunde nötig.\\

            \item \emph{Amazon Prime Video:}\\\\ Zum Streamen von Inhalten nutzt Amazon Video die Microsoft-Silverlight-Technologie
            und die Infrastruktur von Akamai. Zur Wiedergabe von HD-Inhalten wird eine Internetverbindung mit mindestens 3.5 MBits/s
            sowie eine HDCP-Unterstützung aller Geräte benötigt. Die Wiedergabe von SD-Inhalten benötigt jedoch mindestens eine 900 kBit/s
            schnelle Verbindung.

        \end{enumerate}
    \end{addmargin}

\end{document}