%%
%% Author: thompson
%% 01.11.17
%%

% Preamble
\documentclass[11pt]{article}

% Packages
\usepackage{a4wide}
\usepackage[utf8]{inputenc}
\usepackage[ngerman]{babel}

\usepackage{enumerate}
\usepackage{scrextend}

% Document
\begin{document}
    \section{Protokolle der Anwendungsschicht}
    Konsultieren Sie sogenannte RFCs der IETF (erklären Sie die Bedeutung dieser
    Abkürzungen) um folgende Fragestellungen zu beantworten:

    \begin{enumerate}[\thesection .1]
    \item Beschreiben Sie die folgenden Protokolle hinsichtlich ihrer Eigenschaften und
wesentlichen Unterschiede: HTTP/0.9, HTTP/1.0, HTTP/1.1, HTTP/2.0.

    \begin{enumerate}[$\diamond$]
        \item \textbf{HTTP/0.9} :: Erstellt 1991.\\ Es ist ein Subset des vollen HTTP-Protokolls, wie wir es heute kennen.\\
        Merkmale:
        \begin{addmargin}[1em]{1em}
            \begin{enumerate}
                \item Kein Austausch von Klientenprofil.
                \item Einfachheit Request/Response-Model
                \item Keine Sessions oder States
                \item Weitverbreitete Nutzung: Request of Data through Browser
                \item Nutzt telnet-Protokolstil via TCP-IP\\
            \end{enumerate}
        \end{addmargin}

        \underline{Beispiel: Verbindungsaufbau}\\
        Der Klient erstellt eine TCP-IP Verbindung zu einem Host via Domänname oder IP und Port-Nummer.
        Wird kein Port definiert, so wird der default-Wert 80 gesetzt.
        Der Server akzeptiert die Verbinung und Datentransfer kann stattfinden.
    \begin{addmargin}[1em]{1em}
        \emph{telnet mailsrv.aau.at 25} ... verbindung zum Mailserver der AAU auf Port 25\\
    \end{addmargin}

        \underline{Beispiel: Request \& Response}\\
        Ein HTTP-Request wird an einen Server gestellt.
        \begin{addmargin}[1em]{1em}
            \emph{"GET http://www.uni-klu.ac.at:80/myapp/index.html"}
        \end{addmargin}
        Der Server überprüft diese Anfrage.
        \begin{addmargin}[1em]{1em}
            Gibt es dieses Dokument?\\
            Darf der User darauf zugreifen?\\
            Ist es ein dynamisch generiertes Dokument?\\
            Ist das GET-Format richtig? ... u.s.w.
        \end{addmargin}
        Der Server sendet anschließend eine Nachricht zurück zum Anfragensteller.\\
        \footnote[1 Vgl.: HTTP-Statuscodes]{Siehe https://de.wikipedia.org/wiki/HTTP-Statuscode}
        \begin{addmargin}[1em]{1em}
            Information - 100++: Continue, Switching Protocols, Processing\\
            Erfolgreich - 200++: OK, Accepted, Non-Authoritative Information,...\\
            Umleitungen - 300++: Moved permanently, See other, Use Proxy, ...\\
            Client-Fehler - 400++: Bad Request, Unauthorized, Forbidden, Not Found, ...\\
            Server-Fehler - 500++: Internal Server Error, Bad Gateway, Service N/A,...
        \end{addmargin}
        Der Browser des Anfragenstellers erhält das Dokument und rendert gemäß .css/.php/... .
        Bei vollständiger Übertragung des abgerufenen Protokolls unterbricht der Server die Verbindung zum Klienten.\\

        \item \textbf{HTTP/1.0} :: Erweitert um 1996
        Spezifiziert im RFC 1945. Dies erweitert das HTTP/0.9 Protokoll um weitere TCP-IP Verbindungen für multiplen Datentransfer.
        Dadurch werden neben Texten des eigentlichen Dokuments eingebettete Bilder, anhand deren Domäne, geladen.\\
        Eine Website welche 5 Bilder beinhaltet besitzt somit 6 Separate TCP-Verbindungen:
        \begin{addmargin}[1em]{1em}
            - 1. Verbindung: Text\\
            - 2. Verbindung: Bild 1\\
            - 3. Verbindung: Bild 2\\
            - ...\\
            - 6. Verbindung: Bild 5\\
        \end{addmargin}

        \item HTTP/1.1
        \item HTTP/2.0

        %%% Sources:
        % https://www.w3.org/Protocols/HTTP/AsImplemented.html
        % WebTech-VO-Unterlagen, 2nd.pdf
        % https://www.thomas-krenn.com/de/wiki/TCP_Port_80_(http)_Zugriff_mit_telnet_%C3%BCberpr%C3%BCfen

    \end{enumerate}

    \item Beschreiben Sie die folgenden Protokolle und wie sie benutzt werden: SMTP,
POP3, IMAP. Geben Sie ein konkretes Beispiel für SMTP an und demonstrieren
Sie dies mit Hilfe von telnet und mailsrv.uni-klu.ac.at.
    \item Beschreiben Sie das DNS-Protokoll.
\end{enumerate}

\end{document}