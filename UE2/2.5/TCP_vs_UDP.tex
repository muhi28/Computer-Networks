%%
%% Author: muhamed
%% 28.10.17
%%

% Preamble
\documentclass[11pt]{article}

% Packages
\usepackage{a4wide}
\usepackage[utf8]{inputenc}
\usepackage[ngerman]{babel}
\usepackage{scrextend}      % Intending
\usepackage{graphicx}
\usepackage{enumerate}

% Document
\begin{document}

    \section{TCP vs. UDP}
    Wie im OSI-Modell ermöglicht die Transportschicht die Kommunikation
    zwischen den Quell- und Zielhosts. Auf dieser Schicht wurden zwei
    Ende-zu-Ende-Protokolle definiert:\\
    - Transport Control Protocol (TCP)\\
    - User Datagram Protocol (UDP)\\

    \begin{enumerate}
        \item \emph{User Datagram Protocol:}
        \begin{addmargin}[1em]{1em}
            Ist ein sogenannter Datagramm-Dienst, bei dem ledigliche
            Datenpakete verschickt werden. Hierbei muss sich das
            Applikationsprotokoll um Dinge wie Fehlerbehandlung, Quittierung kümmern.
        \end{addmargin}
        \item \emph{Transport Control Protocol:}
        \begin{addmargin}[1em]{1em}
            TCP ist dagegen ein gesichertes, verbindungsorientiertes Protokoll:
            man öffnet zuerst eine "Verbindung" über das dann die Daten übertragen
            werden. Hierbei wird garantiert, dass die Daten vollständig, unverfälscht
            und in der richtigen Reihenfolge ankommen.
            Außerdem funktioniert diese Verbindung in beide Richtungen.
            Anders als bei UDP kann man damt z.B. "Anfragen und Antworten" leicht zuordnen,
            weil sie über dieselbe Verbindung geschickt werden.\\

        \end{addmargin}
    \end{enumerate}

    Die Anwendungsprotokolle sind grundsätzlich für beide Arten dieselben.
    \begin{addmargin}[1em]{1em}
        \begin{enumerate}[$\diamond$]
            \item DNS - Domain Name Service
            \item Internet Control Message Protocol (ICMP)
            \item Telnet
            \item File Transfer Protocol (FTP, TFTP)
            \item HTTP
        \end{enumerate}
    \end{addmargin}
\end{document}