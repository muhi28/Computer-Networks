%%
%% Author: thompson
%% 09.11.17
%%

% Preamble
\documentclass[11pt]{article}

% Packages
\usepackage{a4wide}
\usepackage{graphicx}

% Document
\begin{document}
    \section{Post Office Protocol 3}
    .. anhand RFC 1939.

    \subsection{Allgemeines zu POP3:}
    \begin{enumerate}

        \item ist ein Kommunikationsprotokoll, um Emails von einem POP-Server abzuholen
        \item Kommunikation erfolgt zwischen einem Email Client und einem Email Server

    \end{enumerate}

    \subsection{Funktionsweise des POP3:}

    \begin{enumerate}

        \item per Fernzugriff werde gespeicherte Emails abgerufen und lokal gespeichert
        \item POP sieht Prinzip der Offline-Verarbeitung von E-Mails vor
        \item Online werden E-Mails vom POP-Server des Clients heruntergeladen
        \item erst nach erfolgreichem und vollständigem Zugriff werden Emails am Server gelöscht
        \item Bearbeitung der Emails erfolgt lokal (offline) ohne Verbindunge zum POP-Server

    \end{enumerate}

    \subsection{Anmeldung bei Mailserver:}
    \begin{enumerate}

        \item ExecServer host startet den POP3-ExecServer,in dem es den TCP Port 110 abhört
        \item möchte ein POP3Client den Service verwenden -> POP3Client erstellt TCP
        Verbindung mit dem ExecServer Host her
        \item nach erfolgreicher Verbindung -> POP3 ExecServer sendet Gruß aus
        \item Session kommt in die AUTHORIZATION State
        \item POP3Client muss sich zuletzt identifizieren

    \end{enumerate}

    \subsection{Abrufen von Nachrichten:}
    %\includegraphics[width=\textwidth]{image_preview.gif}
    \begin{enumerate}

        \item POP3Client führt hierbei einen Command aus -> ExecServer antwortet auf dieses
        \item Nachrichten werden von ExecServer lokal am Computer abgespeichert, d.h. es
        wird einfach eine Kopie der Nachricht am Computer abgespeichert, wobei diese
        Mail am ExecServer gelöscht wird und nur mehr lokal erreichbar ist

    \end{enumerate}
    \subsection{Wie ist das POP3-Protokoll grundsätzlich aufgebaut?}
    The Post Office Protocol - Version 3 (POP3) is intended to
    permit a workstation to dynamically access a maildrop on a server
    host in a useful fashion.  Usually, this means that the POP3 protocol
    is used to allow a workstation to retrieve mail that the server is
    holding for it. POP3 is not intended to provide extensive manipulation operations of
    mail on the server - normally, mail is downloaded and then deleted.

    Initially, the server host starts the POP3 service by listening on
    TCP port 110. When a client host wishes to make use of the service,
    it establishes a TCP connection with the server host.  When the
    connection is established, the POP3 server sends a greeting.  The
    client and POP3 server then exchange commands and responses
    (respectively) until the connection is closed or aborted.

    Commands in the POP3 consist of a case-insensitive keyword, possibly
    followed by one or more arguments.  All commands are terminated by a
    CRLF pair.  Keywords and arguments consist of printable ASCII
    characters.  Keywords and arguments are each separated by a single
    SPACE character.  Keywords are three or four characters long. Each
    argument may be up to 40 characters long.

    Responses in the POP3 consist of a status indicator and a keyword
    possibly followed by additional information.  All responses are
    terminated by a CRLF pair.  Responses may be up to 512 characters
    long, including the terminating CRLF.  There are currently two status
    indicators: positive ("+OK") and negative ("-ERR").  Servers MUST
    send the "+OK" and "-ERR" in upper case. Responses to certain commands are multi-line.

    Sessions progresses through a number of states during its
    lifetime.  Once the TCP connection has been opened and the POP3
    server has sent the greeting, the session enters the AUTHORIZATION
    state.  In this state, the client must identify itself to the POP3
    server.  Once the client has successfully done this, the server
    acquires resources associated with the client's maildrop, and the
    session enters the TRANSACTION state.  In this state, the client
    requests actions on the part of the POP3 server.  When the client has
    issued the QUIT command, the session enters the UPDATE state.  In
    this state, the POP3 server releases any resources acquired during
    the TRANSACTION state and says goodbye.  The TCP connection is then
    closed.


    \begin{enumerate}

        \item Starten des POP3 Servers
        \item Indentifizieren des Users innerhalb der AUTHORIZATION State:
        \begin{enumerate}
            \item User muss Benutzernamen sowie Passwort eingeben -> hat ExecServer
            Daten geprüft und ein entsprechendes Maildrop geöffnet ,wechselt POP3
            Session in die TRANSACTION State
            \item POP3Client kann nun Kommandos ausführen, auf welche POP3 ExecServer antwortet

        \end{enumerate}

        \item TRANSACTION State:
        \begin{enumerate}
            \item nachdem POP3Client erfolgreicht identifiziert wurde, kann POP3Client kommandos ausführen
            \item nach jedem Kommando gibt der ExecServer eine Antwort
            \item ist die Sitzung zur Anforderung und Übermittlung der Emails
            \item alle Befehle zur Bearbeitung von Emails werden hier ausgeführt
        \end{enumerate}

        \item UPDATE State:
        \begin{enumerate}
            \item alle vom POP3Client angegebenen Anforderungen werden hier ausgeführt
            \item Verbindung zu TCP ist zu diesem Zeitpunkt schon beendet
            \item führt der POP3Client innerhalb dieser State den QUIT Befehl aus, so
            wechselt die Session zur UPDATE State
            \item terminiert die Session ohne das POP3Client QUIT ausgeführt hat, so kommt
            die Session nicht in den UPDATE Zustand und weiters muss keine Nachricht
            aus dem Maildrop entfernt werden
            \item letzter Vorgang stellt sicher, dass Emails nur auf dem ExecServer gelöscht werden,
            wenn Verbindung ordnungsgemäß beendet wurde
        \end{enumerate}
    \end{enumerate}
    \subsection{Ist POP3 ein sicheres Protokoll? Warum, Warum nicht?}

    \subsection{Was ist der Unterschied zwischen single-line und multi-line response?}

    \subsection{Vergleiche POP3 mit IMAP und beschreibe wesentliche Unterschiede.}

    \begin{enumerate}
        \item POP3:
        \begin{enumerate}
            \item hier werden ledigliche die Emails aus dem Ordner des Posteingangs
            vom ExecServer heruntergeladen
            \item Nutzer kann selbst wählen, ob diese vom ExecServer gelöscht oder behalten
            werden sollen
            \item Meldet man sich an einem anderen Ort an, kann es sein, dass all ihre Emails
            erneut heruntergeladen werden, da diese nicht gelöscht werden
            (kann nach einiger Zeit in die Tausende gehen und viel Speicherplatz/Zeit beanspruchen)
            \item nicht erkennbar, welche Emails gelesen, beantwortet oder gelöscht wurden
            \item Synchronisation zwischen Endgerät und Email-Konto geschieht nicht !!

        \end{enumerate}

        \item IMAP:
        \begin{enumerate}
            \item hierüber wird der komplette Inhalt des Email Kontos stets mit dem Mail-Programm
            auf dem Computer oder Smartphone synchronisiert
            \item wird z.B. eine Nachricht mittels Outlook gesendet, so landet diese im Ordner "Gesendet"
            sowohl in Outlook, als auch auf dem ExecServer und anderen Geräten wie dem Smartphone
            \item alle Bewegungen des Email-Kontos sind auf allen Geräten gleich

        \end{enumerate}
    \end{enumerate}

\end{document}